\documentclass[a4paper,11pt]{scrartcl}
\usepackage{mmap}
\usepackage[margin=1in]{geometry} % decreases margins
\usepackage{setspace}
\setlength{\parskip}{0pt}
\onehalfspacing

\usepackage{enumitem}
\setlist{noitemsep, topsep=0pt, partopsep=0pt, leftmargin=*}

\usepackage{textcomp}
\usepackage{booktabs}
\usepackage[mathrm=sym]{unicode-math}

\usepackage[final]{hyperref} % adds hyper links inside the generated pdf file
\hypersetup{
  colorlinks=true,% false: boxed links; true: colored links
  linkcolor=blue,        % color of internal links
  citecolor=blue,   % color of links to bibliography
  filecolor=magenta,     % color of file links
  urlcolor=blue
}
\urlstyle{rm}

\usepackage[procnames]{listings}
\usepackage[dvipsnames]{xcolor}

\makeatletter
\lstdefinestyle{mystyle}{
  language=python,
  showstringspaces=false,
  otherkeywords={__eq__,__setitem__,__abs__,wrong_int,catch},
  basicstyle=%
    \ttfamily
    \lst@ifdisplaystyle\normalsize\fi,
  keywordstyle=\color{Blue},
  commentstyle=\color[gray]{0.6},
  stringstyle=\color[RGB]{233,125,44},
  procnamekeys={class},
  procnamestyle=\color{Bittersweet},
  emph={None,True,False},
  emphstyle=\itshape\color[rgb]{0.7,0,0},
  emph={[2]self},
  emphstyle=[2]\itshape\color{Bittersweet}
}
\makeatother

\lstset{style=mystyle}
\lstset{escapeinside={(*@}{@*)}}

\usepackage{fontspec}
\usepackage{bold-extra}
\setmonofont[AutoFakeSlant=0.2,Scale=0.95]{D2Coding}
\setsansfont[BoldFont=AppleSDGothicNeo-SemiBold]{Apple SD Gothic Neo}
\setmainfont[AutoFakeSlant=0.2,BoldFont=SDMyeongjoNeoa-eSm,WordSpace={1.0,0.5,0.5},Kerning=On]{SDMyeongjoNeoa-bLt}

\usepackage{kotex}

\addtokomafont{labelinglabel}{\bfseries}
\addtokomafont{title}{\bfseries}

\setkomafont{disposition}{\normalfont}
\setkomafont{section}{\LARGE\bfseries\sffamily}
\setkomafont{subsection}{\Large\mdseries\sffamily}

\title{\vspace{-0.5in}LangComp HW4}
\author{\vspace{-15pt}2016-19986 정누리}
\date{\vspace{-5pt}\today}

%++++++++++++++++++++++++++++++++++++++++

\begin{document}

\maketitle

\begin{labeling}{Q10}
  \item[Q1]
  여기저기 코드를 복사할 필요가 없어지고, 디버깅하기 편하다. 또 프로그램이 짧아지며, 읽기 쉬워지고 수정하기도 쉬워진다.

  \item[Q2]
  When it gets called.

  \item[Q3]
  \lstinline{def} statement.

  \item[Q4]
  A function is an object declared by a \lstinline{def} statement. A function call is the evaluation of associated code block with the function, with the given parameters, to the return value.

  \item[Q5]
  One global scope, many local scopes.

  \item[Q6]
  They are all destroyed.

  \item[Q7]
  A return value is the value that a function call evaluates to. The return value is indeed a value, and an expression consists of values and operators, so it could be used with expression.

  \item[Q8]
  \lstinline{None}.

  \item[Q9]
  With a \lstinline{global} statement.

  \item[Q10]
  \lstinline{NoneType}, and it is the only object of type \lstinline{NoneType}.

  \item[Q11]
  It imports a python module and makes the module available in the scope.

  \item[Q12]
  \lstinline{spam.bacon()}.

  \item[Q13]
  With the \lstinline{try}/\lstinline{catch} statement.

  \item[Q14]
  The code might raise an exception goes in the \lstinline{try} clause. The \lstinline{except} clause will be excecuted once the exception, specified after the \lstinline{except} keyword, was actually raised inside the \lstinline{try} clause.

\end{labeling}
\end{document}
