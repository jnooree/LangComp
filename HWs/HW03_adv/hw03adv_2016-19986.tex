\documentclass[a4paper,11pt]{scrartcl}
\usepackage[margin=1in]{geometry} % decreases margins
\usepackage{setspace}
\setlength{\parskip}{0pt}
\onehalfspacing

\usepackage[procnames]{listingsutf8}
\usepackage[dvipsnames]{xcolor}

\definecolor{stringcolor}{RGB}{233,125,44}

\makeatletter
\lstdefinestyle{mystyle}{
  language=python,
  showstringspaces=false,
  otherkeywords={__eq__,__setitem__,__abs__,wrong_int},
  basicstyle=%
    \ttfamily
    \lst@ifdisplaystyle\normalsize\fi,
  numberstyle=\ttfamily\small\color{gray},
  keywordstyle=\color{Blue},
  commentstyle=\color[gray]{0.6},
  stringstyle=\color{stringcolor},
  procnamekeys={class},
  procnamestyle=\color{Bittersweet},
  emph={None,True,False},
  emphstyle=\itshape\color[rgb]{0.7,0,0},
  emph={[2]self},
  emphstyle=[2]\itshape\color{Bittersweet},
  numbers=left,
  texcl=true
}
\makeatother

\lstset{style=mystyle}
\lstset{escapeinside={(*@}{@*)}}

\usepackage{fontspec}
\usepackage{bold-extra}
\setmonofont[AutoFakeSlant=0.2,Scale=0.95]{D2Coding}
\setsansfont[BoldFont=AppleSDGothicNeo-SemiBold]{Apple SD Gothic Neo}
\setmainfont[AutoFakeSlant=0.2,BoldFont=SDMyeongjoNeoa-eSm,WordSpace={1.0,0.5,0.5},Kerning=On]{SDMyeongjoNeoa-bLt}

\usepackage{kotex}

\addtokomafont{labelinglabel}{\bfseries}
\addtokomafont{title}{\bfseries}

\setkomafont{disposition}{\normalfont}
\setkomafont{section}{\LARGE\bfseries\sffamily}
\setkomafont{subsection}{\Large\bfseries\sffamily}

\renewcommand{\thesubsection}{\arabic{subsection}.}

\title{\vspace{-0.5in}LangComp HW3 Advanced}
\author{\vspace{-15pt}2016-19986 정누리}
\date{\vspace{-5pt}\today}

%++++++++++++++++++++++++++++++++++++++++

\begin{document}

\maketitle

\section*{Code}
\begin{lstlisting}
import re

days = "(*@\textcolor{stringcolor}{월화수목금토일}@*)"
onlychar = re.compile(r"[a-z(*@\textcolor{stringcolor}{ㄱ}@*)-(*@\textcolor{stringcolor}{ㅎㅏ}@*)-(*@\textcolor{stringcolor}{ㅣ가}@*)-(*@\textcolor{stringcolor}{힣}@*)]+")

if __name__ == "__main__":
    # 1. 요일 계산 프로그램
    today = days.index(input("(*@\textcolor{stringcolor}{오늘의 요일을 입력하세요}@*): "))
    N = int(input("(*@\textcolor{stringcolor}{며칠 후의 요일을 계산할까요}@*)? "))
    day = days[(today + N) % 7]
    print(f"{N}(*@\textcolor{stringcolor}{일 후는}@*) {day}(*@\textcolor{stringcolor}{요일입니다}@*).")

    # 2. 단어 빈도 계산 프로그램
    sent = input("(*@\textcolor{stringcolor}{문장을 입력하세요}@*): ")

    freq = dict()
    for match in onlychar.finditer(sent.lower()):
        word = match.group()
        freq[word] = freq.get(word, 0) + 1

    print(freq)

\end{lstlisting}

\newpage
\section*{Test}
\subsection{요일 계산 프로그램}

\begin{labeling}{Case N}
  \item[Case 1]
  {\ttfamily
    오늘의 요일을 입력하세요: 월 \\
    며칠 후의 요일을 계산할까요? 1 \\
    1일 후는 화요일입니다.
  }

  \item[Case 2]
  {\ttfamily
    오늘의 요일을 입력하세요: 화 \\
    며칠 후의 요일을 계산할까요? 10 \\
    10일 후는 금요일입니다.
  }
\end{labeling}

\subsection{단어 빈도 계산 프로그램}

\begin{labeling}{Case N}
  \item[Case 1]
  {\ttfamily
    문장을 입력하세요: Bacon bacon spam, egg bacon!! \\
    \{'bacon': 3, 'spam': 1, 'egg': 1\}
  }

  \item[Case 2]
  {\ttfamily
    문장을 입력하세요: 언컴 컴언? '언컴' SPAm Bacon EGg eGG! \\
    \{'언컴': 2, '컴언': 1, 'spam': 1, 'bacon': 1, 'egg': 2\}
  }

\end{labeling}

\end{document}
